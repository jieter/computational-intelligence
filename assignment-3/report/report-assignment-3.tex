\documentclass[a4paper,10pt,fleqn]{article}
\usepackage{fullpage}
\setlength{\mathindent}{0pt}

% do not indent first line of paragraph.
\setlength{\parindent}{0cm}
% But separate them with 3-10mm
\setlength{\parskip}{6mm plus4mm minus3mm}

\usepackage[dutch]{babel}
\usepackage[utf8]{inputenc}

\usepackage{amsmath}
\usepackage{amssymb}
%~ \usepackage{tabularx}

% Hyperref package, clickable internal links.
% colorlinks to remove ugly boxes around links...
% \usepackage[colorlinks]{hyperref}

\makeatletter
\newcommand*{\centerfloat}{%
  \parindent \z@
  \leftskip \z@ \@plus 1fil \@minus \textwidth
  \rightskip\leftskip
  \parfillskip \z@skip}
\makeatother

\usepackage{enumerate}

\usepackage{subfigure}
\usepackage{pgf}
\usepackage{tikz}
\usetikzlibrary{arrows,automata, shapes}

\usepackage{todonotes}

\title{TI2736-A Assignment 3:}

\author{
    David Akkerman - 4220390 \\
    Jan Pieter Waagmeester - 1222848 \\
}

\begin{document}
\maketitle

\section*{3.1: }
\begin{enumerate}[1.]
	% Make a list of the features you can expect a maze might have. Features that increase the difficulty of ’finding the finish’. Features that will require creative solutions. For example; loops. Name at least 2 other features.
	\item

	% 2. Give an equation for the amount of pheromone dropped by the ants. Answer the questions "Why do we drop pheromone?" and "What is the purpose of the algorithm?".
	\item

	% 3. Give an equation for the evaporation. How much pheromone will vaporise every iteration? This equation should contain variables which you can use to optimize your algorithm. What is the purpose of pheromone evaporation?
	\item

	% 4. Give a short pseudo-code of your ant-algorithm at this stage. If you added any extra functionality to the normal algorithm, please mention and explain them briefly.
	\item

	% 5. Improve the ant algorithm using your own insight. Describe how you improved the standard algorithm; which problems are you tackling and how? Limit yourself to 1 A4 text (excluding aiding figures).
	\item

	% 6. Your task is to find a decent set of parameters for each of the grading mazes. You may do so by varying the parameters and subsequently running your algorithm. If your algorithm converges fast to a good route, your parameters are decent. What is ‘converging fast’? Figure that out by varying the parameters.
	% Report your tactic upon how to vary the parameters. Assist your text with graphs showing relationships between the parameters and the speed of convergence. Limit yourself to 12 A4 text (excluding aiding graphs - and we like nice informative graphs, so use them!).
	\item

	% 7. Using your answer to the previous question, can you say something about the dependency of the parameters on the maze size / complexity? Aim at ≈ 14 A4.
	\item

\end{enumerate}

\end{document}
