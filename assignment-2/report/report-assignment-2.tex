\documentclass[a4paper,10pt,fleqn]{article}
\usepackage{fullpage}
\setlength{\mathindent}{0pt}

% do not indent first line of paragraph.
\setlength{\parindent}{0cm}
% But separate them with 3-10mm
\setlength{\parskip}{6mm plus4mm minus3mm}

\usepackage[dutch]{babel}
\usepackage[utf8]{inputenc}

\usepackage{amsmath}
\usepackage{amssymb}
%~ \usepackage{tabularx}

% Hyperref package, clickable internal links.
% colorlinks to remove ugly boxes around links...
% \usepackage[colorlinks]{hyperref}

\makeatletter
\newcommand*{\centerfloat}{%
  \parindent \z@
  \leftskip \z@ \@plus 1fil \@minus \textwidth
  \rightskip\leftskip
  \parfillskip \z@skip}
\makeatother

\usepackage{enumerate}

% \usepackage{subfigure}
% \usepackage{pgf}
% \usepackage{tikz}
% \usetikzlibrary{arrows,automata}

\title{TI2736-A Assignment 2:  Bayesan network}

\author{
	David Akkerman - 4220390 \\
	Jan Pieter Waagmeester - 1222848 \\
}

\begin{document}
\maketitle


$$ p(A | B) = \frac{p(A \wedge B)}{p(B)}$$

$$ p(A \wedge B) = p(A|B) \cdot p(A)$$
$$ p(A \wedge B) = p(B|A) \cdot p(B)$$

Als onafhankelijk:
$$P(A, B) = P(A) \cdot P(B)$$

Marginalisatie (verwijderen van variabele):
 $$P(A, B) = \sum_C P(A, B, C)$$

\section*{2.1: A simple network I}
\begin{enumerate}[1.]
	% Compute the chance that the customer wants to purchase fruit (i.e., ‘P (F )’). Please write down an exact equation for P (F ) before filling it in. This will help you a lot for the later questions.
	\item $$P(F) = \sum_{\forall i} \sum_{\forall j} (P(F|V_iH_j) \cdot P(V_i) \cdot P(H_j)) = 0.29$$

	% 2. Compute the chance that the customer wants to purchase candy. Again, start with the exact equation for P (C ).
	\item $$P(C) = \sum_{\forall i} \sum_{\forall j} (P(C|H_iK_j) \cdot P(H_i) \cdot P(K_j)) = 0.42 $$

	% Now assume that our customer is a vegetarian: “evidence is found that our customer is a vegetarian”.

	% 3. Compute the chance that this customer wants to purchase fruit.
	\item $$P(F|V) = \sum_{\forall i} P(F|VH_i) \cdot P(H_i) = 0.79	$$

	% 4. And what is the chance that the vegetarian customer wants to purchase candy?
	\item $$P(C|V) = P(C) = 0.42$$

	% 5. Does the fact that the customer is a vegetarian change our belief in the customer having healthy eating habits? I.e., does P (H ) change, now that we known that V is true?
	% Please give your answer using a maximum of 30 words.
	\item Nee, voor de berekeningen is aangenomen dat $V$ en $H$ onafhankelijk zijn. Dit betekent dus dat geldt: $$P(H|V) = P(H)$$ Dit betekent dat $P(H)$ niet verandert als $V$ bekend is.

    % 6. What is the chance that this customer wants to purchase fruit?
    \item $$P(F|V\lnot H) = 0.7$$

    % 7. And what is the chance for this customer to purchase candy?
    \item $$P(C|V\lnot H) = P(C|\lnot H) = \sum_{\forall i} P(C|\lnot HK_i) \cdot P(K_i) = 0.58 	$$

    % 8. What is the chance that this customer wants to purchase fruit?
    \item $$P(F|\lnot VH \lnot K) = P(F|\lnot VH) = 0.9	$$

    % 9. And what is the chance for this customer to purchase candy?
    \item $$P(C|\lnot VH\lnot K) = P(C|H\lnot K) = 0.01	$$

\end{enumerate}

\section*{2.2: A simple network II}

\begin{enumerate}[1.]
	\setcounter{enumi}{10}

    % 10. In the lecture you have seen Bayes’ rule, e.g. in the “cold” example. Use Bayes’ rule to derive an expression for P(V |M). Express your answer using only known odds (Eqs.2.1 - 2.11, Eqs. 2.12 - 2.13). If done correctly, your equation should be of the form P(V |M) = a/a+b
    \item $$P(M) = \frac{P(M|V)\cdot P(V)}{P(M)} = \frac{P(M|V)\cdot P(V)}{P(M|V)\cdot P(V) + P(M|\lnot V)\cdot P(\lnot V)} = 4.4 \cdot 10^{-5}  $$
    $$P(M) = \sum_{\forall i} P(M|V_i) \cdot (V_i) = 0.91  $$

    % 11. Compute the chance that the customer is a vegetarian, given that he does not want to purchase any meat (P(V |¬M)).
    \item $$P(V|\lnot M) = \frac{P(\lnot M|V) \cdot P(V)}{P(\lnot M)} = 0.11	$$

    % 12. A similar, but yet more complicated, equation may be written down to compute P(V |F). The only real difference is the fact that you will need an extra elimination of variables.
	\item

    $$ P(V|F) =$$

	% 13. Compute the chance that the customer is a vegetarian, given that he does want to purchase fruit (P (V |F )).
    \item
    $$ P(V| F) = $$

    % 14. Now, write down an equation for P (V |M F ). If done correctly, your equation should be of the form P (V |M F ) = (a+c)/(a+c+b+d)

    \item $$P(V |M, F) = \frac{a + c}{a+c+b+d}$$

	% 15. Compute the chance that the customer is a vegetarian, given that he does want to purchase fruit, but no meat (P (V |F ¬M )).
    \item $$P(V |F, \neg M)$$


\end{enumerate}

\section*{2.3: A non-boolean network}
\begin{enumerate}[1.]
	\setcounter{enumi}{16}
    % 16. Write down an equation for P (H ) using the terms listed in Eqs. 2.14 - 2.18.
    \item $$ P(H) = $$

    % 17. Compute P (H ) and P (¬H ).
    \item

    % 18. Use the knowledge acquired in As. 2.2 to compute the chance that the customer is an inactive sporter, given that the customer has healthy eating habbits.
    \item

\end{enumerate}

\section*{2.4: A network for our robot}
\begin{enumerate}[1.]
	\setcounter{enumi}{19}
    % 19. Give a schematic diagram of the network relevant to the above text. Show all the proper variables and causalities. (Hint: There are 5 variables in total)
    \item

    % 20. write down all the conditional- and a-priori probabilities.
    \item

    % 21. Expand the network with the given probabilities. Make a diagram of the nodes with arrows indicating the direction of influence (H = Healthy, M = Married, C = Children, V = Vegetarian, F = Female).
    \item

    % 22. Make a screenshot of your network for in the report.
    \item

    % 23. What’s the probability the customer will want to buy products from class 5?
    \item

    % 24. What’s the probability the customer will not want anything from class 2?
    \item

    % 25. Given that the customer is a female. What is the probability of having to buy something in class 4?
    \item

    % 26. This same woman turns out to have two children. What is the chance of her wanting something from class 4 now?
    \item

    % 27. A different customer profiles as a healthy vegetarian. What’s the probability of buying a product from class 3?
    \item

    % 28. A customer was seen buying something out of class 2. What is the probability that he/she is married?
    \item

    % 29. What’s the chance this customer is female?
	\item

    % 30. A wealthy man was observed to buy products from all categories. What is the probability that he’s is married?
    \item

    % 31. Another customer appeared to not buy anything from the listed classes. What can you say about this customer?
    \item

    % 32. We know that a certain person bought stuff from class 1 and 2, but we did not observe the rest of his basket. Is this person likely to have any children?
    \item

    % 33. We do not know what a given person bought, but we are certain he didn’t buy anything from class 2, 3 or 5. What can you say about this customer?
    \item
\end{enumerate}

\end{document}
