\documentclass[a4paper,10pt,fleqn]{article}
\usepackage{fullpage}
\setlength{\mathindent}{0pt}

% do not indent first line of paragraph.
\setlength{\parindent}{0cm}
% But separate them with 3-10mm
\setlength{\parskip}{6mm plus4mm minus3mm}

\usepackage[dutch]{babel}
\usepackage[utf8]{inputenc}

%~ \usepackage{fullpage}
%~ \usepackage[official]{eurosym} % depens on: texlive-fonts-recommended
\usepackage{amsmath}
\usepackage{amssymb}
%~ \usepackage{tabularx}

% Hyperref package, clickable internal links.
% colorlinks to remove ugly boxes around links...
% \usepackage[colorlinks]{hyperref}

\usepackage{enumerate}

\usepackage{subfigure}
\usepackage{pgf}
\usepackage{tikz}
\usetikzlibrary{arrows,automata}

\title{TI2736-A\\ Assignment 1:  Artificial Neural Network}

\author{
	Arthur Hovanesyan - \\
	David Akkerman - \\
	Jan Pieter Waagmeester - 1222848 \\
}

\begin{document}
\maketitle

\begin{enumerate}[1.]
	\item Omdat er 10 eigenschappen zijn hebben we 10 inputneurons nodig.
	\item
	\item
	\item
	\item Ons netwerk kan voorgesteld worden als in dit plaatje: \\

	\begin{center}
		\tikzstyle{neuron}=[draw=black!50,minimum size=20pt,inner sep=3pt]
		\tikzstyle{input}=[rectangle]
		\tikzstyle{hidden}=[circle]
		\tikzstyle{output}=[circle]
		\begin{tikzpicture}[node distance=3cm,scale=1,auto]
			\def \scalar {0.9}

			\def \inputs {10}
			\def \hidden {7}
			\def \outputs {7}


			\foreach \m in {1,...,\inputs}{
				\node[neuron,input] (input-\m) at(0, {(-\m + \inputs / 2) * \scalar}) {$\m$};
			}

			\foreach \m in {1,...,\hidden}{
				\node[neuron,hidden] (hidden-\m) at(7, {(-\m + \hidden / 2) * \scalar}) {$\m$};
			}

			\foreach \m in {1,...,\outputs}{
				\node[neuron,output] (output-\m) at(14, {(-\m + \outputs / 2) * \scalar}) {$\m$};
			}

			\foreach \i in {1,...,\inputs}{
				\foreach \h in {1,...,\hidden}{
					\path (input-\i) edge [draw=black!40] (hidden-\h);
				}
			}

			\foreach \h in {1,...,\hidden}{
				\foreach \o in {1,...,\outputs}{
					\path (hidden-\h) edge [draw=black!40] (output-\o);
				}
			}

			\node (input) at (0, {(-\inputs / 2 - 1) * \scalar}) {inputs};
			\node (hidden) at (7, {(-\hidden / 2 - 1) * \scalar}) {hidden};
			\node (output) at (14, {(-\outputs / 2 - 1) * \scalar}) {outputs};
		\end{tikzpicture}
	\end{center}
\end{enumerate}
\end{document}

